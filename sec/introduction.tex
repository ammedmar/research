\section*{Introduction} \label{s:introduction}

There is a tense trade-off in algebraic topology having roots reaching back to the beginning of its modern form.
This tension can be illustrated with the concept of cohomology.
The first approaches, dating back to Poincar\'e, are based on the subdivision of a space into simple contractible pieces.
These elementary shapes are made to generate a free graded module and their spatial relations define the differential used to compute cohomology.
This definition makes certain geometric properties of cohomology, for example excision, fairly clear.
Yet, it is not easy to show that a continuous map of spaces induces a map between their associated cohomologies.
The functoriality just alluded to is trivial when defining cohomology in terms of homotopy classes of maps to Eilenberg--MacLane spaces, but the passage to the homotopy category erases geometric and combinatorial information and the resulting definition is not well suited for concretely presented spaces.
The tension this example illustrates manifests itself in many other important contexts, and the trade-off between effectiveness and functoriality remains as central today as it was almost a century ago.

A unifying goal of this research program is to ease said tension by developing concrete constructions of concepts defined only indirectly and, by doing so, allowing for the application of some of the central ideas of algebraic topology in novel contexts, most notably data analysis and topological field theory.
A central concept in this program, which is motivated next, is that of commutativity up to coherent homotopies.

Algebraic topology uses tools from abstract algebra to study topological spaces.
Prominently, by creating algebraic invariants of spaces up to homotopy equivalence.
A first example of an algebraization procedure is provided by the functor of simplicial or cubical cochains.
This construction may be thought of as a linearization of spaces and, as expected, it losses much structure.
For example, the cohomology groups of $\mathbb{C} P^2$ and $S^2 \vee S^4$, the union of a $2$- and a $4$-sphere over a point, are isomorphic despite these spaces not being equivalent.
More information can be encoded on the quasi-isomorphism type of the cochain complex of a space if equipped with a natural product structure.
This makes its cohomology into a natural commutative ring.
In our example, $\mathbb{C} P^2$ and $S^2 \vee S^4$ are indeed distinguished by their cohomology rings.
However, this invariant has noticeable limitations.
For example, taking the suspension of these leads to two non-homotopic spaces $\Sigma(\mathbb{C} P^2)$ and $\Sigma(S^2 \vee S^4)$ whose cohomologies are isomorphic as graded rings.
Further structure on cohomology can be used to distinguish them.
For example, Steenrod operations in the mod $p$ cohomology of spaces, which together with the Bockstein homomorphism provide a complete account of the mod $p$ cohomology functor.
To define said operations, one constructs a coherent family of homotopies correcting the broken commutativity of the cochain level product.
This ``derived commutativity structure'' fully faithfully encodes the entire homotopy type of spaces with certain finiteness conditions.
Its existence can be argued indirectly, but this misses the rich geometric and combinatorial structures that an effective construction unveils, and provides no method for the computation of the associated invariants for concretely presented spaces.

\section*{Organization}

\cref{s:past} presents an overview of the significance of these homotopy coherent structures and this program's contributions to this and other areas of mathematics, including topological data analysis, higher category theory, functional analysis, and information theory.
\cref{s:current} is devoted to discussing research projects that are currently being pursued in this framework, touching on convex geometry, topological field theories, and knot theory.
This note concludes presenting in \cref{s:future} an overview of the two application domains in the title; data analysis and field theories, and how this program will continue making contributions to them.
