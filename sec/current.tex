\section{Current work} \label{s:current}

\subsection{Stability of Steenrod barcodes}

Their \textit{stability} is an important reason for the widespread use of the barcode invariant in both applied and theoretical mathematics.
This property states roughly that with respect to the natural distance of barcodes, the passage from filtered spaces with the Gromov--Hausdorff distance to barcodes is Lipschitz.
In joint work with F. M\'emoli and L. Zhou, a stability result for Steenrod barcodes is proven and some specific Lipschitz bounds for certain continuous and discrete examples are established.

\subsection{Multisimplicial chains and configuration spaces}

Following a suggestion of P.~May, in joint work with A.~Pizzi and P.~Salvatore, the model discussed in \cref{ss:e-infty operads} is used to define a natural $E_\infty$-structure on multisimplicial chains.
This is used to construct a zig-zag of quasi-isomorphisms of $E_\infty$-coalgebras between the singular chains of euclidean configuration spaces and the multisimplicial models of McClure--Smith.

\subsection{Framed polytopes, higher categories and Steenrod diagonals} \label{ss:polytopes}

In joint work with G. Laplante-Anfossi and B. Vallette an exploration of the convex geometry of higher diagonals of polytopes is presented.
The uncovered structures create a bridge between Steenrod type constructions and higher category theory.
More specifically, it is shown that both; a higher diagonal on a $n$-dimensional polytope $P$ and a free $n$-category structures on it can be canonically constructed from a generic frame of $P$.
The techniques developed are used to provide a full proof of a theorem of Kapranov and Voevodsky announced in \cite{kapranov1991polycategories}.

\subsection{Odd prime Steenrod operations: persistence theory and Khovanov homology} \label{ss:odd prime steenrod operations}

In joint work with F. Cantero-Mor\'an, \emph{stable} cup-$(p,i)$ products of cochains are introduced.
There are three sources of motivation for this construction.
The first is to develop chain approximations to generalized homology theories extending those defined by Brumfiel--Morgan and discussed in more detail in \cref{ss:spt phases}.
The second is to generalize the constructions of Steenrod squares on Khovanov homology presented Cantero-Mor\'an in \cite{cantero-moran2020khovanov}.
The third is to obtain faster algorithms to compute Steenrod operations of finite simplicial complexes, facilitating the potential incorporation of this additional structure into topological data analysis as in \cref{ss:steenroder}.


\subsection{Geometric cohomology} \label{ss:geometric cohomology}

de Rham cohomology has long been lauded as a perfect cohomology theory by champions such as Sullivan and Bott.
A combination of geometric underpinning and commutativity at the cochain level make it a remarkably effective tool for many applications of rational homotopy theory.
Over the integers, submanifolds and intersection in various settings provide geometrically meaningful cochains with a partially defined commutative product.
Joint work with D. Sinha and G. Friedman provides a firm foundation for a model of ordinary (co)homology of smooth manifolds based on maps from manifolds with corners as generating (co)chains.
The first application of this theory already appeared in \cite{medina2021flowing} and was summarized above in \cref{ss:flowing}.
Several more applications of this theory are to follow.
Eventually, it is expected to geometrically encode the $E_\infty$-structure of manifolds in a geometric form.

\subsection{Non-invertible Dijkgraaf--Witten theory}

In traditional Dijkgraaf--Witten theory the fields are principal bundles for a finite
group $G$, which form a groupoid.
An alternative, but homotopy equivalent, description of the collection of all fields on a $d$-manifold $M$ is the mapping space $\mathrm{Map}(M, \rB G)$, which is an $\infty$-groupoid.
Joint work with L. M\"uller and L. Stehouwer moves away from the manifest invertibility of these groupoids by considering a triangulation on $M$ and the free $n$-category structure it generates -- please compare with \cref{ss:nerve} and \cref{ss:polytopes}.
The resulting more general theories have fields given by $d$-functors from $M$ to appropriate $d$-categorical generalizations of $\rB G$.