\section{Current work} \label{s:current}

\subsection{Stability of Steenrod barcodes}

Their \textit{stability} is an important reason for the widespread use of the barcode invariant of filtered spaces in both applied and theoretical mathematics.
This property states roughly that with respect to the natural distance of barcodes, the passage from filtered spaces with the Gromov--Hausdorff distance to barcodes is Lipschitz.
In joint work with F. M\'emoli and L. Zhou, we prove a stability result for Steenrod barcodes and explore some specific Lipschitz bounds for certain continuous and discrete examples.

\subsection{Multisimplicial chains and configuration spaces}

Following a suggestion of P.~May, with A.~Pizzi and P.~Salvatore, we use my model of the $E_\infty$-operad to define a natural $E_\infty$-structure on the chains of multisimplicial sets.
We use this to construct a zig-zag of quasi-isomorphisms of $E_\infty$-coalgebras between the singular simplicial chains of the configuration space of~$k$ points in $\R^n$, and the \textit{surjections} multisimplicial model of McClure--Smith.

\subsection{Framed polytopes, higher categories and Steenrod diagonals} \label{ss:polytopes}

Together with G. Laplante-Anfossi and B. Vallette we explore the convex geometry of higher diagonals of polytopes.
The uncovered structures create a bridge between Steenrod type constructions and higher category theory.
More specifically, we show a connection between higher diagonals on a $n$-dimensional polytope $P$ and free $n$-category structures on it.
Both being canonically constructed from a generic frame for $P$.
Additionally, we use our techniques to provide a proof of a theorem of Kapranov and Voevodsky announced in \cite{kapranov1991polycategories}.

\subsection{Odd prime Steenrod operations: persistence theory and Khovanov homology} \label{ss:odd prime steenrod operations}

With F. Cantero-Mor\'an we introduce cup-$(p,i)$ products that are stable at the cochain level.
There are two sources of motivation for this result.
On one hand, for applications in persistent homology or computational topology more broadly, this description leads to faster algorithms computing Steenrod operations of finite simplicial complexes.
On the other hand, we will use them to generalize the constructions of Steenrod squares on Kovanov homology presented in \cite{cantero-moran2020khovanov} to all primes.
We also plan to implement algorithms resulting from our formulas for the concrete study of knots and links.

\subsection{Geometric cohomology} \label{ss:geometric cohomology}

de Rham cohomology has long been lauded as a perfect cohomology theory by champions such as Sullivan and Bott.
A combination of geometric underpinning and commutativity at the cochain level make it a remarkably effective tool for many applications of rational homotopy theory.
Over the integers, submanifolds and intersection in various settings provide geometrically meaningful cochains with a partially defined commutative product.
As part of a long term project with D. Sinha and G. Friedman we are working to provide a firm foundation for a model of ordinary (co)homology of smooth manifolds based on maps from manifolds with corners as generating (co)chains.
The first application of this theory already appeared in \cite{medina2021flowing} and was summarized above in \cref{ss:flowing}.
Several more applications are to follow.
Eventually we expect to encode geometrically the $E_\infty$-structure of manifolds in a geometric form using our theory.

\subsection{Non-invertible Dijkgraaf--Witten theory}

In traditional Dijkgraaf--Witten theory the fields are principal bundles for a finite
group $G$, which form a groupoid.
An alternative, but homotopy equivalent, description of the collection of all fields on a $d$-manifold $M$ is the mapping space $\mathrm{Map}(M, \rB G)$, which is an $\infty$-groupoid.
With L. Muller and L. Stehouwer, we move away from the manifest invertibility of these groupoids by
considering a triangulation on $M$ and the free $n$-category structure it generates -- please compare with \cref{ss:nerve} and \cref{ss:polytopes}.
The resulting more general theories have fields given by $d$-functors from $M$ to appropriate $d$-categorical generalizations of $\rB G$.