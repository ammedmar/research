% !TEX root = ../theoretical.tex

\section{Previous work} \label{s:past}

\subsection{Steenrod cup-$i$ construction} \label{ss:cup-i}

In the late thirties, Alexander and Whitney defined the ring structure on cohomology introducing a cochain level construction dual to a choice of simplicial chain approximation to the diagonal map.
For any space $X$ this is a natural chain map
\[
\Delta_0 \colon \chains(X) \to \chains(X) \otimes \chains(X)
\]
where $\chains(X)$ denotes the integral singular chains of $X$.
Unlike the diagonal of spaces, this chain map is not symmetric with respect to transpositions of factors in the target.
Steenrod \cite{steenrod1947products} corrected homotopically the broken symmetry of $\Delta_0$ by effectively constructing maps
\begin{equation*}
\Delta_i \colon \chains(X) \to \chains(X) \otimes \chains(X)
\end{equation*}
satisfying certain homological conditions.
These so called \textit{cup-$i$ coproducts}, or their dual \textit{cup-$i$ products}, define mod $2$ cohomology operations known as \textit{Steenrod squares} whose importance in stable homotopy theory is hard to overstate.

In \cite{medina2023fast_sq}, I introduced new formulas defining cup-$i$ coproducts and use them to provide a much faster algorithm for the computation of Steenrod squares on simplicial complexes.

\subsection{Steenrod barcodes}

Persistent homology is a central tool in the rapidly growing field of topological data analysis.
We used the new description of the cup-$i$ products and associated algorithm for the computation of Steenrod squares to effectively incorporate into the persistence pipeline the finer information encoded by these cohomology operations.
With other members of \giottoTDA's team (\cref{ss:giotto}) we develop a high-performance implementation of these methods in the package \texttt{steenroder}, which we have used to identify fine topological features of real-world data \cite{medina2022per_st}.

\subsection{Topological data analysis and machine learning} \label{ss:giotto}

I am part of the team that develops the open-source software \texttt{giotto-tda}, a \texttt{Python} library that integrates high-performance topological data analysis with machine learning via a \emph{scikit-learn}--compatible API and state-of-the-art \texttt{C++} implementations \cite{medina2021giotto}.
We are adding to this versatile multi-task package the functionalities of \texttt{steenroder}, making easier to machine learning practitioner to use the above enhancements of the traditional barcode.
The \giottoTDA\ project is the result of a public-private partnership between the \href{https://www.epfl.ch/labs/hessbellwald-lab/}{\textit{Laboratory for Topology and Neuroscience}} at EPFL, the machine learning company \href{https://www.l2f.ch/}{\textit{L2F SA}}, and the \href{https://heig-vd.ch/en/research/reds}{\textit{Institute of Reconfigurable \& Embedded Digital Systems}} of HEIG-VD.

\subsection{Axioms}

In principle, the cup-$i$ coproducts introduced by Steenrod and those of \cite{medina2023fast_sq} are only homotopic.
Yet, they happen to be isomorphic.
This is also true of all other known cup-$i$ constructions as consequence of the following result.
In \cite{medina2022axiomatic}, I showed that similarly to how Steenrod squares operations can be characterized axiomatically, any collection of higher products realizing the ``derived commutativity" of the mod $2$ cohomology algebra is isomorphic to Steenrod's if it satisfies certain natural axioms.

\subsection{Nerve of higher categories} \label{ss:nerve}

The ubiquitous nature of the cup-$i$ products is explained by their intrinsic grounding on the combinatorics of simplices.
The next result shows this by deriving from the cup-$i$ products another fundamental construction: the nerve of higher categories.
A $0$-category is a set and an $(n+1)$-category is a category enriched over $n$-categories.
In \cite{medina2020globular}, I constructed a functor from cup-$i$ coalgebras to $n$-categories.
Then, I showed that the complex of chains on the standard $n$-simplex $\simplex^n$ endowed with Steenrod cup-$i$ coproducts is sent to the free $n$-category generated by $\simplex^n$.
These $n$-categories are regarded as ``fundamental structures of nature" (\cite{street1987orientals} pag. 289) and define the nerve of $n$-categories by a standard procedure.

\subsection{Cartan} \label{ss:cartan}

Steenrod squares $Sq^k$ and the algebra structure on cohomology are related via the Cartan formula
\begin{equation*}
Sq^k(\alpha \beta) = \sum_{i+j=k} Sq^i(\alpha) Sq^j(\beta).
\end{equation*}
Since both of these structures are induced from the cup-$i$ products it is natural to wonder if a proof at the cochain level can be given, ideally one that constructs a cochain whose coboundary is the difference between cochains representing the relation.
This is what I did in \cite{medina2020cartan}.
Furthermore, the proof is effective enough that I was able to write an open-source computer implementation of the construction.
The original motivation for this project and the one below came from physicist Anton Kapustin.
More on the connection to physics in (\cref{ss:spt phases}).

\subsection{Adem} \label{ss:adem}

The iteration of the $Sq^k$ operations satisfy the Adem relations
\begin{equation*}
Sq^i Sq^j(\alpha) = \sum_{k=0}^{\lfloor i/2 \rfloor} \binom{j-k-1}{i-2k} Sq^{i+j-k} Sq^k(\alpha)
\end{equation*}
for all $i,j > 0$ such that $i < 2j$.
These relations play an important role in the definition of secondary cohomology operations and $\kappa$-invariants of Postnikov towers.
In joint work with G. Brumfiel and J. Morgan \cite{medina2021adem} we provided a construction of cochains enforcing these relations in the same spirit as those I constructed for the Cartan relation.

\subsection{Ranicki--Weiss Assembly}

In \cite{medina2022assembly}, I considered the assembly functor of Ranicki--Weiss \cite{ranicki1990assembly} from chain complex valued presheaves on a simplicial complex $X$ to chain complexes, which can be extended to the L-theory assembly functors of \cite{ranicki1992topological} by considering chain complexes with derived Poincar\'e duality.
Ranicki and Weiss showed that assembly lifts to a functor to comodules over the Alexander--Whitney coalgebra of $X$.
In this paper, I showed that assembly lifts to comodules over the Steenrod cup-$i$ coalgebra of $X$ and, more importantly, that this lift is fully faithful.

%\subsection{Algebraic representations}
%
%In \cite{medina2020globular}, I constructed a functor from globular sets, a higher dimensional generalization of directed graphs, to coalgebras with higher coproducts proving it defines a full and faithful embedding.
%This result was inspired by my thesis, where I did the same for simplicial complexes and proved additionally that the category of representations of the poset of simplices, i.e., functors from this category to chain complexes, embeds fully faithfully into the category of comodules over the associated coalgebra with higher coproducts.
%This result allowed me to reinterpret the total surgery obstruction of Ranicki \cite{ranicki1992topological} in terms of cobordism classes of these comodules enriched with a derived notion of duality \cite{medina2015thesis}.

%As mentioned in \cref{fast_sq}, Steenrod introduced his square operations in the mod~2 cohomology of spaces via explicit formulas for cup-$i$ products of cochain.
%Indirectly, these operations can be defined using the mod~2 homology of the symmetric group $\sym_2$ and have analogues for any prime $p$ using the mod~$p$ homology of $\sym_p$.
%Using work by May one can unify these constructions integrally using $E_\infty$-algebras, but no formulas were known analogous to the cup-$i$ products for odd Steenrod operations.
%In this paper we present such formulas for algebras over certain models of the $E_\infty$-operad, and derive from them cup-$(p,i)$ products for simplicial and, using work described in \cref{cube_einfty}, also cubical sets.

\subsection{Steenrod operations} \label{ss:may steenrod}

So far we have focused on operations in mod 2 cohomology.
The definition of Steenrod operations for odd primes was given in non-constructive terms using the homology of symmetric groups, and analogues of the cup-$i$ products were missing from the literature.
In \cite{medina2021may_st}, using the operadic viewpoint of May, we introduced --~for any prime $p$~-- cup-$(p,i)$ products defining constructively all Steenrod operations for simplicial and, using work described in \cref{ss:e-infty structures}, also cubical sets.
These constructions are explicit enough that I implemented them in the computer algebra system \texttt{comch} \cite{medina2021comch}, an open source project I develop to manipulate models of the $E_\infty$-operad, a concept I now describe.

%\subsection{C.A.S. \texttt{ComCH}}
%
%I wrote the specialized computer algebra system \texttt{ComCH} to study and manipulate combinatorial models of the $E_\infty$-operad and their action on cellular (co)chains \cite{medina2021comch}.
%To date, this lightweight \texttt{Python} project models the Barratt-Eccles and surjection operads \cite{mcclure2003multivariable,berger2004combinatorial}, and implements Steenrod cup-$(p,i)$ products for simplicial and cubical sets as defined in \cite{medina2021may_st}.

\subsection{\pdfEinfty-operads}

The cup-$i$ products are part of a more general structure known as an \textit{$E_\infty$-algebra}, a notion controlled by so called \textit{$E_\infty$-operads}.
The study of $E_\infty$-structures has a long history, where (co)homology operations, the recognition of infinite loop spaces, and the complete algebraic representation of the $p$-adic homotopy category are key milestones.
No finitely presented $E_\infty$-operad can exist but, as I showed in \cite{medina2020prop1,medina2021prop2}, passing to the context of multiple inputs and outputs allows for the introduction of props whose associated operads are $E_\infty$.
I related my models in the categories of chains complexes and CW-spaces to $E_\infty$-operads previously defined by McClure--Smith, Berger--Fresse and R.~Kaufmann, and used these comparisons to establish a conjecture of the latter.

\subsection{\pdfEinfty-structures} \label{ss:e-infty structures}

Given its small number of generators and relations, my model of the $E_\infty$-operad is well suited to define $E_{\infty}$-structures concretely.
In \cite{medina2020prop1}, I defined an $E_\infty$-algebra structure on simplicial cochains extending the Alexander--Whitney product and, in \cite{medina2022cube_einfty}, one on cubical cochains extending the Cartan product.
Additionally, we showed that the Cartan--Serre comparison map from simplicial to cubical singular chains of spaces is a quasi-isomorphism of $E_\infty$-algebras.

\subsection{Adams' cobar construction}

In the fifties, Adams introduced a comparison map from his cobar construction on the (simplicial) singular chains of a pointed space to the cubical singular chains on its based loop space.
This comparison map is a quasi-isomorphism of algebras, which was shown by Baues to be one of bialgebras by considering Serre's cubical coproduct.
With M. Rivera in \cite{medina2021cobar}, we used my model of the $E_\infty$-operad to generalize Baues result proving that Adams' comparison map is a quasi-isomorphism of $E_{\infty}$-bialgebras, i.e. of monoids in the category of $E_{\infty}$-coalgebras.

\subsection{The diagonal of cellular spaces}

For these spaces, an $E_\infty$-structure can be thought of as arising from coherently correcting up to homotopies the broken symmetry of the diagonal.
When this is done over the rationals, one obtains the notion of $C_\infty$-coalgebra which controls rational homotopy theory.
In \cite{medina2022dennis}, written in the occasion of Dennis Sullivan's $80^\th$ birthday, I presented a survey of effective algebraic constructions and open problems in the theory of $C_\infty$ and $E_\infty$-structures on cellular spaces.

\subsection{Flowing from intersection product to cup product} \label{ss:flowing}

Effective constructions are not only relevant in algebraic contexts.
In \cite{medina2021flowing} we took a geometric viewpoint and use a vector field flow, defined through a cubulation of a closed manifold, to reconcile -- at the cochain level -- the partially defined commutative product on geometric cochains with the standard cup product on cubical cochains, which is fully defined and commutative only up to coherent homotopies. The interplay between intersection and cup product dates back to the beginnings of homology theory, but, to our knowledge, this result is the first to give an explicit cochain level comparison between these approaches.

\subsection{Persistence theory and functional topology}

Persistence homology is a powerful tool in applied topology which is now having impact in theoretical mathematics as well.
In \cite{medina2022fuct_top}, we introduced topological conditions on a broad class of functionals that ensure that the persistent homology modules of their associated sublevel set filtration admit persistence diagrams, which, in particular, implies that they satisfy generalized Morse inequalities. We illustrate the applicability of these results by recasting the original proof of the Unstable Minimal Surface Theorem given by Morse and Tompkins in a modern and rigorous framework.

\subsection{Information theory}

In \cite{medina2021hyperharmonic}, we used a discrete version of the Laplace-de Rham operator to study information theory.
More precisely, generalizations, from pairs to tuples, of Shannon's mutual information.
In this paper we combined methods from harmonic analysis and combinatorial topology to construct efficient representations of high-order information-theoretic signals, and showed in a real-world example that these representations are highly compressible.
This addresses the problem of exponential growth of these high-order information signals.
