\section{Previous work} \label{s:past}

\subsection{Steenrod cup-$i$ construction} \label{ss:cup-i}

In the late thirties, Alexander and Whitney defined the ring structure on cohomology introducing a cochain level construction dual to a choice of simplicial chain approximation to the diagonal map.
For any space $X$ this is a natural chain map
\[
\Delta_0 \colon \chains(X) \to \chains(X) \otimes \chains(X)
\]
where $\chains(X)$ denotes the integral singular chains of $X$.
Unlike the diagonal of spaces, this chain map is not symmetric with respect to transpositions of factors in the target.
Steenrod \cite{steenrod1947products} corrected homotopically the broken symmetry of $\Delta_0$ by effectively constructing maps
\begin{equation*}
\Delta_i \colon \chains(X) \to \chains(X) \otimes \chains(X)
\end{equation*}
satisfying certain homological conditions.
These so called \textit{cup-$i$ coproducts}, or their dual \textit{cup-$i$ products}, define mod $2$ cohomology operations known as \textit{Steenrod squares} whose importance in stable homotopy theory is hard to overstate.

New formulas defining cup-$i$ coproducts were introduced in \cite{medina2023fast_sq} and used to provide a much faster algorithm for the computation of Steenrod squares on simplicial complexes.

\subsection{Steenrod barcodes} \label{ss:steenroder}

Persistent homology is a central tool in the rapidly growing field of topological data analysis.
In \cite{medina2022per_st}, the new description of the cup-$i$ products and associated algorithm for the computation of Steenrod squares were used to effectively incorporate into the persistence pipeline the finer information encoded by these cohomology operations.
With other members of \giottoTDA's team (\cref{ss:giotto}) a high-performance implementation of these methods was developed in the package \href{https://steenroder.github.io/steenroder/}{\texttt{steenroder}}, which was used to detect Steenrod barcodes in real-world data \cite{medina2022per_st}.

\subsection{Topological data analysis and machine learning} \label{ss:giotto}

In order to make topological data analysis accessible to machine learning practitioners, the open-source software \href{https://giotto-ai.github.io/gtda-docs/latest/library.html}{\texttt{giotto-tda}} was developed \cite{medina2021giotto}.
It is currently one of the most used tools in this area.
The \giottoTDA\ project, is the result of a public-private partnership supported by the Suisse innovation agency Innosuisse.

\subsection{Axioms}

The formulas introduced by Steenrod and the author give rise to homotopic cup-$i$ coproducts, yet the question of there exact agreement remained open.
Similarly to how Steenrod squares can be characterized axiomatically, the cup-$i$ products can be as well \cite{medina2022axiomatic}.
In fact, this work shows that all cup-$i$ products formulas in the literature give rise to constructions isomorphic to Steenrod's original.

\subsection{Nerve of higher categories} \label{ss:nerve}

The ubiquitous nature of the cup-$i$ products is explained by their intrinsic grounding on the combinatorics of simplices.
The next result shows this by deriving from the cup-$i$ products another fundamental construction: the nerve of higher categories.
A $0$-category is a set and an $(n+1)$-category is a category enriched over $n$-categories.
In \cite{medina2020globular}, a functor from cup-$i$ coalgebras to $n$-categories is constructed.
It takes the complex of chains on the standard $n$-simplex $\simplex^n$, endowed with Steenrod cup-$i$ coproducts, to the free $n$-category generated by $\simplex^n$.
These $n$-categories are regarded as ``fundamental structures of nature" (\cite{street1987orientals} pag. 289) and define the nerve of $n$-categories by a standard procedure.

\subsection{Cartan} \label{ss:cartan}

Steenrod squares $\Sq^k$ and the algebra structure on cohomology are related via the Cartan formula:
\begin{equation*}
\Sq^k(\alpha \beta) = \sum_{i+j=k} \Sq^i(\alpha) \Sq^j(\beta).
\end{equation*}
Since both of these structures are induced from the cup-$i$ products it is natural to wonder if a proof at the cochain level can be given, ideally one that constructs a cochain whose coboundary is the difference between cochains representing the relation.
This was done in \cite{medina2020cartan}, and the proof is effective enough that an open-source implementation of the construction was developed.
The original motivation for this project and the one below came from physicist Anton Kapustin.
More on the connection to physics in (\cref{ss:spt phases}).

\subsection{Adem} \label{ss:adem}

The iteration of the $\Sq^k$ operations are related by the Adem formula:
\begin{equation*}
\Sq^i \Sq^j(\alpha) = \sum_{k=0}^{\lfloor i/2 \rfloor} \binom{j-k-1}{i-2k} \Sq^{i+j-k} \Sq^k(\alpha)
\end{equation*}
for all $i,j > 0$ such that $i < 2j$.
These relations play an important role in the definition of secondary cohomology operations and $\kappa$-invariants of Postnikov towers.
In the same spirit as (\cref{ss:cartan}), in joint work with G. Brumfiel and J. Morgan \cite{medina2021adem} an effective construction of cochains enforcing these relations was given.

\subsection{Ranicki--Weiss assembly}

The assembly functor of Ranicki--Weiss \cite{ranicki1990assembly} goes from chain complex valued presheaves on a simplicial complex $X$ to chain complexes.
It can be extended to the L-theory assembly functors of \cite{ranicki1992topological} by considering chain complexes with derived Poincar\'e duality.
Ranicki and Weiss showed that assembly lifts to comodules over the Alexander--Whitney coalgebra of $X$.
In \cite{medina2022assembly}, it is shown that assembly lifts to comodules over the Steenrod cup-$i$ coalgebra of $X$ and, more importantly, that this lift is fully faithful.

\subsection{Steenrod operations} \label{ss:may steenrod}

So far this note has focused on operations in mod 2 cohomology.
The definition of Steenrod operations for odd primes was given in non-constructive terms using the homology of symmetric groups, and analogues of the cup-$i$ products were missing from the literature.
Using the operadic viewpoint of May, cup-$(p,i)$ products defining constructively all Steenrod operations
were introduced in \cite{medina2021may_st}.
These constructions are explicit enough to have been implemented in the computer algebra system \href{https://comch.readthedocs.io/en/latest/}{\texttt{comch}} \cite{medina2021comch}, an open source project developed to manipulate $E_\infty$-structures, a concept described next.

\subsection{\pdfEinfty-operads}

The cup-$i$ products are part of a more general structure known as an \textit{$E_\infty$-algebra}, a notion controlled by so called \textit{$E_\infty$-operads}.
After work of M. Mandell \cite{mandell2006cochains}, it is known that the $E_\infty$-algebra of a space satisfying suitable finiteness conditions encodes its entire homotopy type.
No finitely presented $E_\infty$-operad can exist but, as shown in \cite{medina2020prop1,medina2021prop2}, passing to the context of multiple inputs and outputs allows for the introduction of props whose associated operads are $E_\infty$.
These new models were related to those previously defined by McClure--Smith, Berger--Fresse and R.~Kaufmann, establishing as a consequence a conjecture of the last author.

\subsection{\pdfEinfty-structures} \label{ss:e-infty structures}

Given its small number of generators and relations, my model of the $E_\infty$-operad is well suited to define $E_{\infty}$-structures concretely.
In \cite{medina2020prop1}, an $E_\infty$-algebra structure on simplicial cochains extending the Alexander--Whitney product was defined, and, in \cite{medina2022cube_einfty}, also one on cubical cochains extending the Cartan product was described.
Additionally, it was shown that the Cartan--Serre comparison map from simplicial to cubical singular chains of spaces is a quasi-isomorphism of $E_\infty$-algebras with respect to these structures.

\subsection{Adams' cobar construction} \label{ss:e-infty operads}

In the fifties, Adams introduced a comparison map from his cobar construction on the (simplicial) singular chains of a pointed space to the cubical singular chains on its based loop space.
This comparison map is a quasi-isomorphism of algebras, which was shown by Baues to be one of bialgebras by considering Serre's cubical coproduct.
In joint work with  M. Rivera \cite{medina2021cobar}, the above models of the $E_\infty$-operad were used generalize Baues result proving that Adams' comparison map is a quasi-isomorphism of $E_{\infty}$-bialgebras, i.e. of monoids in the category of $E_{\infty}$-coalgebras.

\subsection{The diagonal of cellular spaces}

For cellular spaces, an $E_\infty$-structure can be thought of as arising from coherently correcting up to homotopies the broken symmetry of the diagonal.
When this is done over the rationals, one obtains the notion of $C_\infty$-coalgebra which controls rational homotopy theory.
The survey \cite{medina2022dennis}, written in the occasion of Dennis Sullivan's $80^\th$ birthday, presents an overview of effective algebro-homotopical constructions and open problems in the theory of $C_\infty$ and $E_\infty$-structures on cellular spaces.

\subsection{Flowing from intersection product to cup product} \label{ss:flowing}

Effective constructions are not only relevant in algebraic contexts.
In joint work with G. Friedman and D. Sinha \cite{medina2021flowing} the geometric viewpoint is taken.
In it, a vector field flow defined through a cubulation of a closed manifold is used to reconcile -- at the cochain level -- the partially defined commutative product on geometric cochains with the standard cup product on cubical cochains, which is fully defined but commutative only up to coherent homotopies.
The interplay between intersection and cup product dates back to the beginnings of homology theory, but, to our knowledge, this result is the first to give an explicit cochain level comparison between these approaches.

\subsection{Persistence theory and functional topology}

Persistence homology is a powerful tool in applied topology which is now having impact in theoretical mathematics as well.
In joint work with U. Bauer and M. Schmahl \cite{medina2022fuct_top}, topological conditions were introduced on a broad class of functionals ensuring their persistent homology admits a persistence diagram, which, in particular, implies that these functionals satisfy generalized Morse inequalities.
We illustrate the applicability of these results by recasting the original proof of the Unstable Minimal Surface Theorem given by Morse and Tompkins in a modern and rigorous framework.

\subsection{Information theory}

Much attention in the study of complex systems has been given to generalizations, from pairs to tuples, of Shannon's mutual information.
Since in general the number of simplices grows exponentially with the number of vertices, finding efficient representations of these high-order information-theoretic signals is a challenge.
In joint work with F. Rosas, S. Rodr\'igez and R. Cofr\'e  \cite{medina2021hyperharmonic}, a combination of methods from harmonic analysis and combinatorial topology is used to construct alterative representations of these signals, which are shown in a real-world example to be highly compressible.