\section{Application domains} \label{s:future}

\subsection{Homotopical algebra, data analysis and AI} \label{ss:ai}

Topology can complement traditional approaches to machine learning and data analysis by providing global summaries of complex relational structures.
The added information is model-independent and highly resistant to noise.
Designing reliable AI systems is a multi-faceted challenge and homotopical algebra can be used to tackle the following core aspects of the problem:

\begin{itemize}
	\item Robustness to noise and adversarial attacks: deep learning models are surprisingly susceptible to small perturbations, and this can lead to unexpected failures in deployed systems.
	\item Generalizability to unseen data: large models can overfit the data they are trained on and thus fail to generalize to data in deployment.
\end{itemize}
Traditional regularizers, one of the primary tools used to tackle the above challenges, impose penalties on representations that exhibit undesirable properties such as exploding gradients or highly complicated geometric structures.
By studying the topology of the weight space and decision boundaries of trained models, we will validate on real-world data the assumption that robustness is linked to less cumbersome topologies.
This will enable to create performant topological regularizers that improve the resilience and generalization power of new models.

To quantify the topological complexity of these spaces and device such regularization procedures, developments from the sub-field of homotopical algebraic data analysis, such as Steenrod persistence modules, are fundamental.
Together with part of \giottoTDA's team, we will continue to develop -- at all stages from theory and algorithms to implementation and deployment -- novel homotopical invariants that can be used for these and similar challenges.
International collaborations are also envision, in particular, with the Laboratory for Topology and Neuroscience at EPFL, where as a member I wrote the mathematical content of a grant on this topic awarded for over 1 million USD.

\subsection{Symmetry protected topological phases and cochain constructions} \label{ss:spt phases}

A central problem in physics is to define and understand the moduli ``space'' of quantum systems with a fixed set of invariants, for example their dimension and symmetry type.
In condensed matter physics, quantum systems are presented using \textit{lattice models} which, intuitively, are given by a Hamiltonian presented as a sum of local terms on a Hilbert space associated to a lattice in $\R^n$.
We think of these as defined on flat space.
One such system is said to be \textit{gapped} if the spectrum of the Hamiltonian is bounded away from $0$, and two Hamiltonians represent the same \textit{phase} if there exists a deformation between them consisting only of systems that remain bounded from below.

Given a lattice model, using cellular decompositions and state sum type constructions one can often compute the corresponding \textit{partition functions} on spacetime manifolds from actions expressed in terms of gauge fields represented by cochains and cochain level structures: Stiefel--Whitney cochains, cup-$(p, i)$ products (\cref{ss:cup-i}, \cref{ss:may steenrod}) and Cartan/Adem coboundaries (\cref{ss:cartan}, \cref{ss:adem}) among others.
Subdivision invariance gives rise to a functorial TQFT, which in the \textit{invertible} case is expected to be controlled by a generalized cohomology theory.
The cochain level structure used in the definition of the cellular gauge theory is interpreted from this point of view as describing a cochain model of the Postnikov tower of the relevant spectrum.
For example, fermionic phases protected by a $G$-symmetry are believed to be classified by applying to $BG$ the Pontryagin dual of spin bordism.
Building on these insights and using a formula of mine \cite{medina2020cartan}, A.~Kapustin proposed a structural ansatz in low dimensions that G. Brumfiel and J. Morgan verified by constructing cochain models of certain connective covers of said spectrum.

In the future, with my collaborators we will continue deepening the understanding of the discrete and algebraic structures underpinning SPT phases, with the ultimate goal of elucidating the effects in physiscs of developing, alongside the functorial approach to stable homotopy theory, the effective one as well.

%between discrete models, stable homotopy theory, topological field theories.
%This is a long term project, but a first concrete objective is the description of the signature of 4 manifolds through a combinatorial local state sum formula.
%The importance of this question has been highlighted by Peter Teichner.
%Partial results for general $4k$-manifolds have been achieved by Dennis Sullivan and Andrew Ranicki \cite{sullivan1976signature}.