\section{Application domains} \label{s:future}

\subsection{Homotopical algebra, data analysis and AI} \label{ss:ai}

Topology can complement traditional approaches to machine learning and data analysis by providing global summaries of complex relational structures.
The added information is model-independent and highly resistant to noise.
Designing reliable AI systems is a multi-faceted challenge and homotopical algebra can be used to tackle the following core aspects of the problem:

\begin{itemize}
	\item Robustness to noise and adversarial attacks: deep learning models are surprisingly susceptible to small perturbations, and this can lead to unexpected failures in deployed systems.
	\item Generalizability to unseen data: large models can overfit the data they are trained on and thus fail to generalize to data in deployment.
\end{itemize}
Traditional regularizers, one of the primary tools used to tackle the above challenges, impose penalties on representations that exhibit undesirable properties such as exploding gradients or highly complicated geometric structures.
By studying the topology of the weight space and decision boundaries of trained models, this research will validate on real-world data the assumption that robustness is linked to less cumbersome topologies; enabling the creation of performant topological regularizers that improve the resilience and generalization power of new models.

To quantify the topological complexity of these spaces and device such regularization procedures, developments from homotopical algebra need be made effective, as done for Betti numbers and Steenrod squares which lead to the usual and Steenrod barcodes.
Together with members of \giottoTDA's team and other collaborators, the author will continue to develop -- at all stages from theory and algorithms to implementation and deployment -- novel homotopical invariants that can be used for these and similar challenges.

\subsection{Symmetry protected topological phases and cochain constructions} \label{ss:spt phases}

A central problem in physics is to define and understand the moduli ``space'' of quantum systems with a fixed set of invariants, for example their dimension and symmetry type.
In condensed matter physics, quantum systems are presented using \textit{lattice models} which, intuitively, are given by a Hamiltonian presented as a sum of local terms on a Hilbert space associated to a lattice in $\R^n$.
We think of these as defined on flat space.
One such system is said to be \textit{gapped} if the spectrum of the Hamiltonian is bounded away from $0$, and two Hamiltonians represent the same \textit{phase} if there exists a deformation between them consisting only of systems that remain bounded from below.

Given a lattice model, by means of cellular decompositions and state sum type constructions, one can often compute the associated \textit{partition functions} on spacetime manifolds.
The fields and actions been expressed using cochain level structure, for example, Stiefel--Whitney cochains, cup-$i$ products and Adem coboundaries.
Subdivision invariance gives rise to a functorial TQFT, which in the \textit{invertible} case is expected to be controlled by a generalized cohomology theory.
The cochain level structure used in the definition of the cellular gauge theory is interpreted from this point of view as describing a cochain model of the Postnikov tower of the relevant spectrum.
For example, fermionic phases protected by a $G$-symmetry are believed to be classified by applying to $BG$ the Pontryagin dual of spin bordism.
Building on these insights and using a formula introduced in \cite{medina2020cartan}, A.~Kapustin proposed a structural ansatz in low dimensions that G. Brumfiel and J. Morgan verified by constructing cochain models of certain connective covers of said spectrum.

In the future, the research program presented here will continue deepening the understanding of the discrete and algebraic structures underpinning SPT phases, with the ultimate goal of elucidating the physics content of the complementary viewpoints provided in stable homotopy theory by functorial and effective constructions.

%of developing, alongside the functorial approach to stable homotopy theory, the effective one as well.

%between discrete models, stable homotopy theory, topological field theories.
%This is a long term project, but a first concrete objective is the description of the signature of 4 manifolds through a combinatorial local state sum formula.
%The importance of this question has been highlighted by Peter Teichner.
%Partial results for general $4k$-manifolds have been achieved by Dennis Sullivan and Andrew Ranicki \cite{sullivan1976signature}.