
\section{Previous work}

\subsection{Steenrod cup-$i$ construction}

In the late thirties, Alexander and Whitney defined the ring structure on cohomology introducing a cochain level construction dual to a choice of simplicial chain approximation to the diagonal map.
For any space $X$ this is a natural chain map
\[
\Delta_0 \colon \chains(X) \to \chains(X) \otimes \chains(X)
\]
where $\chains(X)$ denotes the integral singular chains of $X$.
Unlike the diagonal of spaces, this chain map is not symmetric with respect to transpositions of factors in the target.
Steenrod \cite{steenrod1947products} corrected homotopically the broken symmetry of $\Delta_0$ by effectively constructing maps
\begin{equation*}
\Delta_i \colon \chains(X) \to \chains(X) \otimes \chains(X)
\end{equation*}
satisfying certain homological conditions.
These so called \textit{cup-$i$ coproducts}, or their dual \textit{cup-$i$ products}, define mod $2$ cohomology operations known as \textit{Steenrod squares} whose importance in stable homotopy theory is hard to overstate.

In \cite{medina2021newformulas}, I gave a new description of cup-$i$ coproducts and use it to provide a much faster algorithm for the computation of Steenrod squares on simplicial complexes.

\subsection{Persistence}

Persistent homology is a central tool in the rapidly growing field of topological data analysis.
The new description of the cup-$i$ products and associated algorithm for the computation of Steenrod squares were used in \cite{medina2018persistence} to effectively incorporate into the persistence pipeline the finer information encoded by these cohomology operations.

With other members of \texttt{giotto-tda}'s team (\cref{ss:giotto}) we develop a high-performance implementation of these methods which we have used to identify fine topological features of real-world data.

\subsection{Axioms}

In principle, the cup-$i$ coproducts introduced by Steenrod \cite{steenrod1947products} and those of \cite{medina2021newformulas} are only homotopic.
Yet, they happen to be isomorphic.
This is also true of all other known cup-$i$ constructions \cite{gonzalez-diaz1999steenrod, mcclure2003multivariable, berger2004combinatorial, medina2020prop1} as consequence of the following result.

In \cite{medina2018axiomatic}, I showed that similarly to how Steenrod squares operations can be characterized axiomatically \cite{steenrod1962cohomology}, any collection of higher products realizing the ``derived commutativity" of the mod $2$ cohomology algebra is isomorphic to Steenrod's if it satisfies certain natural axioms.

\subsection{Higher categories}

The ubiquitous nature of these formulas is explained by their intrinsic grounding on the combinatorics of simplices.
The next result shows this by deriving from the cup-$i$ products another fundamental construction: the nerve of higher categories \cite{street1987orientals}.

A $0$-category is a set and an $(n+1)$-category is a category enriched over $n$-categories.
In \cite{medina2020globular}, I constructed a functor from coalgebras with higher coproducts to $n$-categories.
Then, I showed that the complex of chains on the standard $n$-simplex $\simplex^n$ endowed with Steenrod cup-$i$ coproducts is sent to the free $n$-category generated by $\simplex^n$.
These $n$-categories are regarded as ``fundamental structures of nature" (\cite{street1987orientals} pag. 289) and define the nerve of $n$-categories by a standard procedure.

\subsection{Algebraic representations}

In \cite{medina2020globular}, I constructed a functor from globular sets, a higher dimensional generalization of directed graphs, to coalgebras with higher coproducts proving it defines a full and faithful embedding.
This result was inspired by my thesis, where I did the same for simplicial complexes and proved additionally that the category of representations of the poset of simplices, i.e., functors from this category to chain complexes, embeds fully faithfully into the category of comodules over the associated coalgebra with higher coproducts.
This result allowed me to reinterpret the total surgery obstruction of Ranicki \cite{ranicki1992topological} in terms of cobordism classes of these comodules \cite{medina2015thesis}.

\subsection{Cartan} \label{ss:cartan}

Steenrod squares $Sq^k$ and the algebra structure on cohomology are related via the Cartan formula
\begin{equation*}
Sq^k(\alpha \beta) = \sum_{i+j=k} Sq^i(\alpha) Sq^j(\beta).
\end{equation*}
Since both of these structures are induced from the cup-$i$ products it is natural to wonder if a proof at the cochain level can be given, ideally one that constructs a cochain whose coboundary is the difference between cochains representing the relation.
This is what I did in \cite{medina2020cartan}.
Furthermore, the proof is effective enough that I was able to write an open-source computer implementation of the construction.
The original motivation for this project and the one below came from physicist Anton Kapustin.
He used my constructions in \cite{kapustin2017fermionic} for the study of topological phases of matter.

\subsection{Adem} \label{ss:adem}

The iteration of the $Sq^k$ operations satisfy the Adem relations
\begin{equation*}
Sq^i Sq^j(\alpha) = \sum_{k=0}^{\lfloor i/2 \rfloor} \binom{j-k-1}{i-2k} Sq^{i+j-k} Sq^k(\alpha)
\end{equation*}
for all $i,j > 0$ such that $i < 2j$.
These relations play an important role in the definition of secondary cohomology operations and $\kappa$-invariants of Postnikov towers.

In joint work with G. Brumfiel and J. Morgan \cite{medina2021adem} we provided a construction of cochains enforcing these relations in the same spirit as those I constructed for the Cartan relation.

\subsection{\pdfEinfty-operads}

The cup-$i$ products are part of a more general structure known as an \textit{$E_\infty$-algebra}, a notion controlled by so called \textit{$E_\infty$-operads}.
The study of $E_\infty$-structures has a long history, where (co)homology operations \cite{steenrod1962cohomology, may1970general}, the recognition of infinite loop spaces \cite{boardman1973homotopy, may1972geometry}, and the complete algebraic representation of the $p$-adic homotopy category \cite{mandell2001padic} are key milestones.

No finitely presented $E_\infty$-operad can exist but, as I showed in \cite{medina2020prop1, medina2018prop2}, passing to the context of multiple inputs and outputs allows for the introduction of props whose associated operads are $E_\infty$.
I related my models in the categories of chains complexes and CW-spaces to $E_\infty$-operads previously defined by McClure--Smith \cite{mcclure2003multivariable}, Berger--Fresse \cite{berger2004combinatorial} and R.~Kaufmann \cite{kaufmann2009dimension}, and used these comparisons to establish a conjecture of his.

\subsection{\pdfEinfty-structures} \label{ss:e-infty structures}

Given its small number of generators and relations, my model of the $E_\infty$-operad is well suited to define $E_{\infty}$-structures.
In \cite{medina2020prop1}, I defined an $E_\infty$-algebra structure on simplicial cochains extending the Alexander--Whitney product and, with Kaufmann in \cite{medina2021cubical}, one on cubical cochains extending the Cartan product.
Showing, additionally, that the Cartan--Serre comparison map from simplicial to cubical singular chains of spaces is a quasi-isomorphism of $E_\infty$-algebras.

\subsection{Adams' cobar construction}

In the fifties, Adams introduced a comparison map from his cobar construction on the (simplicial) singular chains of a pointed space to the cubical singular chains on its based loop space \cite{adams1956cobar}.
This comparison map is a quasi-isomorphism of algebras, which was shown by Baues to be one of bialgebras by considering Serre's cubical coproduct \cite{baues1998hopf}.
With M. Rivera in \cite{medina2021cobar}, we generalized Baues result proving that Adams' comparison map is a quasi-isomorphism of $E_{\infty}$-bialgebras, i.e. of monoids in the category of $E_{\infty}$-coalgebras.
We also show that Hess--Tonks' extended cobar construction \cite{hess2010cobar} on the chains of a general reduced simplicial set is quasi-isomorphic as an $E_{\infty}$-coalgebra to the chains on the associated Kan loop group.

\subsection{Steenrod operations} \label{ss:may steenrod}

The definition of Steenrod operations for odd primes was given in non-constructive terms through the homology of symmetric groups \cite{steenrod1952reduced, steenrod1962cohomology}.
In \cite{medina2020maysteenrod}, using the operadic viewpoint of May \cite{may1970general}, we generalized Steenrod's cup-$i$ products to cup-$(p,i)$ multivariate products of cochains, defining constructively all Steenrod operations for simplicial and cubical sets, as well as for the cobar construction applied to the chains on a reduced simplicial set.

\subsection{C.A.S. \texttt{ComCH}}

I wrote the specialized computer algebra system \texttt{ComCH} to study and manipulate combinatorial models of the $E_\infty$-operad and their action on cellular (co)chains \cite{medina2021computer}.
To date, this lightweight \texttt{Python} project models the Barratt-Eccles and surjection operads \cite{mcclure2003multivariable,berger2004combinatorial}, and implements Steenrod cup-$(p,i)$ products for simplicial and cubical sets as defined in \cite{medina2020maysteenrod}.

\subsection{Geometric cohomology} \label{ss:flows}

As part of a long term project with D. Sinha and G. Friedman we are working to provide a firm foundation for a model of ordinary (co)homology of smooth manifolds based on maps from manifolds with corners as generating (co)chains (\cref{ss:foundations}).
The first application of this theory appeared in \cite{medina2021flowing}, where we use a vector field flow defined through a cubulation of a closed manifold to reconcile the partially defined commutative product on geometric cochains with the standard cup product on cubical cochains, which is fully defined and commutative only up to coherent homotopies.

\subsection{Functional topology}

During the 1930s, Marston Morse developed a vast generalization of what is commonly known as Morse theory relating the critical points of a semi-continuous functional with the topology of its sublevel sets.
Morse and Tompkins applied this body of work, referred to as functional topology, to prove the Unstable Minimal Surface Theorem.
Several concepts introduced by Morse in this context can be seen as early precursors to the theory of persistent homology.
With U. Bauer and M. Schmahl, we provided a modern redevelopment of the homological aspects of Morse's functional topology from the perspective of persistence theory \cite{medina2021functional}.
We adjusted several key definitions and proved stronger statements, including a generalized version of the Morse inequalities.
As an application, we identified and corrected a mistake in the proof of the Unstable Minimal Surface Theorem by Morse and Tompkins.

\subsection{Information theory}

In \cite{medina2021hyperharmonic}, we use a discrete version of the Laplace-de Rham operator to study information theory.
More precisely, generalizations, from pairs to tuples, of Shannon's mutual information.
We combined methods from harmonic analysis and combinatorial topology to construct efficient representations of high-order information-theoretic signals, and showed in a real-world example that these faithful representations are highly compressible.

\subsection{Topological data analysis} \label{ss:giotto}

I am part of the team that develops the open-source software \texttt{giotto-tda}, a \texttt{Python} library that integrates high-performance topological data analysis with machine learning via a \emph{scikit-learn}--compatible API and state-of-the-art \texttt{C++} implementations \cite{medina2021giotto}.