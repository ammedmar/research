
\section{Steenrod squares and cup-$i$ products}

In the late thirties, Alexander and Whitney, building on the work of \v{C}ech and Kolmogorov, defined the ring structure on cohomology, the cup product,
\begin{equation*}
[\alpha] [\beta] = [\alpha \smallsmile_{0} \beta]
\end{equation*}
using a cochain level construction
\begin{equation*}
\smallsmile_0 \, \colon N^*(X; \mathbb Z) \otimes N^*(X; \mathbb Z) \to N^*(X; \mathbb Z)
\end{equation*}
dual to a choice of simplicial chain approximation to the diagonal inclusion.
Here $N^*(X; \mathbb Z)$ denotes the integral singular cochains of a space $X$.

Steenrod \cite{steenrod1947products} then showed that the broken symmetry of $\smallsmile_0$ can be homotopically corrected by effectively constructing products
\begin{equation*}
\smallsmile_i\, \colon N^*(X; \mathbb Z) \otimes N^*(X; \mathbb Z) \to N^*(X; \mathbb Z)
\end{equation*}
realizing its ``derived commutativity".
These higher order products, known as cup-$i$, lead to the cohomology operations
\begin{equation*}
Sq^k \colon H^*(X; \Ftwo) \to H^{*}(X; \Ftwo)
\end{equation*}
that lie at the heart of stable homotopy theory.

In \cite{medina2021newformulas}, I introduced a new description of the cup-$i$ products, in terms of the




Alexander and Whitney defined the cup product by dualizing a chain approximation to the diagonal:
\[
\gchains(\gsimplex^n) \to \gchains(\gsimplex^n) \otimes \gchains(\gsimplex^n).
\]
Similarly, Cartan and Serre constructed: $\gchains(\gcube^n) \to \gchains(\gcube^n) \otimes \gchains(\gcube^n)$.

As mentioned before, as graded rings,
\[
H^\bullet(\mathbb{C} P^2) \not\cong H^\bullet(S^2 \vee S^4).
\]

\[
H^\bullet(\Sigma(\mathbb{C} P^2)) \cong H^\bullet(\Sigma(S^2 \vee S^4)),
\]
where $\Sigma$ denotes suspension, for example $\Sigma(S^1)$ is
\begin{center}
	\includegraphics[scale=.2]{aux/suspension.pdf}
\end{center}

These chain approximations, unlike the diagonal of spaces, are not invariant under transposition: $x \otimes y \stackrel{T}{\mapsto} y \otimes x$.

By correcting homotopically the breaking of this symmetry, Steenrod defined further structure on mod 2 cohomology:
\[
Sq^k \colon H^\bullet(X; \Ftwo) \to H^\bullet(X; \Ftwo).
\]

Distinguishes
\[
H^\bullet(\Sigma(\mathbb{C} P^2)) \not\cong H^\bullet(\Sigma(S^2 \vee S^4))
\]

Steenrod introduced his corrections by effectively constructing
\[
\cW(2) \otimes \gchains(\gsimplex^n) \to \gchains(\gsimplex^n) \otimes \gchains(\gsimplex^n)
\]
an equivariant chain map, where $\cW(2)$ is the minimal free resolution of $\Z$ as an $\Z[\S_2]$-module:
\[
\Z[\S_2]\{e_0\} \xla{1-T} \Z[\S_2]\{e_1\} \xla{1+T} \dotsb
\]

One such map replacing $\cW(2)$ by $\Z$ gives a symmetric chain approx.

Think of $\cW(2)$ as a ``equivariant derived version" of $\Z$.

The orbits of the elements $e_i$ represent the homology of $\rB \S_2 \sim \R P^\infty$.

The map $\Delta_i \colon \gchains(\gsimplex^n) \to \gchains(\gsimplex^n)^{\otimes 2}$ defined by $e_i$ is called \textit{cup-$i$ coproduct}.


\[
d_u[v_0, \dots, v_m] = [v_0, \dots, \widehat v_u, \dots, v_m]
\]
\[
\rP_q(n) = \big\{ U \subseteq \{0,\dots,n\} : \bars{U} = q \big\}
\]
\[
\forall \, U = \{u_1 < \dots < u_q\} \in \rP_q(n)
\]
\[
d_U = d_{u_1} \dotsm \, d_{u_q}
\]
\[
U^\varepsilon = \big\{ u_i \in U \mid u_i + i \equiv \varepsilon \text{ mod } 2 \big\}
\]
\begin{definition}[Med.]
	For a basis element $x \in \gchains_m(\gsimplex^n)$
	\vspace*{-5pt}
	\[
	\Delta_i(x) \ = \!\!\! \sum_{U \in \rP_{m-i}(n)} \!\! d_{U^0}(x) \otimes d_{U^1}(x)
	\]
	\vspace*{-10pt}
\end{definition}

\begin{align*}
\Delta_0 [0,1,2] &=
\Big( d_{12} \otimes \id + d_2 \otimes d_1 + \id \otimes d_{01} \Big) [0,1,2]^{\otimes 2} \\ &=
[0] \otimes [0,1,2] + [0,1] \otimes [1,2] + [0,1,2] \otimes [2].
\end{align*}










\newpage
\ \\
\newpage

\subsection{An axiomatic characterization of Steenrod's cup-$i$ products}

In this section we survey the main result of \cite{medina2018axiomatic}.
It states that any collection of higher products realizing the derived commutativity of the cohomology algebra is isomorphic to Steenrod's original cup-$i$ products if it is natural, minimal, non-degenerate, and free.

Let $\Sigma_2$ be the group with one non-identity element $T$ and let
\begin{equation} \label{e:definition of W}
W = \big( \Ftwo[\Sigma_2] \stackrel{1+T}{\longleftarrow} \Ftwo[\Sigma_2] \stackrel{1+T}{\longleftarrow} \cdots \big)
\end{equation}
be the minimal free resolution of $\Ftwo$ as an $\Ftwo[\Sigma_2]$-module.
We denote the preferred element in degree $i$ by $e_i$.

A \textbf{symmetric product} on a differential graded $\Ftwo$-module $C$ is a chain map
\begin{equation*}
W \tensor_{\Sigma_2} C^{\tensor 2} \to C
\end{equation*}
where $T$ acts by multiplication on $W$ and by transposition on $C \tensor C$.
We denote the image of $[e_i \tensor \alpha \tensor \beta]$ by $\alpha \smallsmile_i \beta$.

A \textbf{symmetric product construction} is a symmetric product on each $N^*(X; \Ftwo)$, where $X$ is a simplicial set, that is natural with respect to simplicial maps.
An \textbf{isomorphism} of symmetric product constructions is an automorphism $\phi$ of $W$ making the diagram
\begin{center}
	\begin{tikzcd}
	W \displaytensor_{\Sigma_2} N^*(X; \Ftwo) \arrow[dr, in=180, out=-90] \arrow[r, "\phi \tensor \id"] & W \displaytensor_{\Sigma_2} N^*(X; \Ftwo) \arrow[d] \\
	& N^*(X; \Ftwo)
	\end{tikzcd}
\end{center}
commute for every simplicial set $X$.

The first axiom alluded to in the beginning of the section, naturality, has been explicitly absorbed into our definition of symmetric product construction; whereas the second, minimality, is manifested in the definition of symmetric product via the use of $W$.
Let us introduce the other two axioms whose definitions reference $N^*(\simplex^d; \Ftwo)$, the normalized cochains on the representable simplicial sets.

We say a symmetric product construction is \textbf{non-degenerate} if for any basis element $\sigma = [v_0, \dots, v_i]$
\begin{equation*}
\boxed{\sigma^* \smallsmile_{i} \sigma^* \neq 0}
\end{equation*}
and we say it is \textbf{free} if for any pair of basis elements $\sigma_j = [v_0, \dots, v_{n_j}]$ for $j = 1, 2$
\begin{equation*}
\boxed{\sigma^*_1 \smallsmile_{i} \sigma^*_2 = \sigma^*_2 \smallsmile_{i} \sigma^*_1}\
\Longrightarrow\
\boxed{\sigma^*_1 \smallsmile_{i} \sigma^*_2 = 0}
\end{equation*}
whenever $i \neq n_1$ or $i \neq n_2$.

\begin{theorem} [\cite{medina2018axiomatic}]
	Up to isomorphism, Steenrod's original cup-$i$ construction \cite{steenrod1947products} is the only free non-degenerate symmetric product construction.
\end{theorem}

We will provide evidence for the fundamental nature of this construction in the next section by relating it to the nerve of higher-dimensional categories.

\subsection{Cup-$i$ products and the nerve of higher categories}

Roberts \cite{roberts1977mathematical} pioneered the idea of using higher-dimensional categories as the coefficient objects for non-abelian cohomology.
A key ingredient for this enterprise is the construction of a nerve functor from \mbox{$n$-categories} to simplicial sets.
Such a functor can be obtained from an $n$-category $\mathcal{O}_n$ naturally assigned to each object $[n]$ in the simplex category $\simplex$.
The construction of these $\mathcal O_n$ was accomplished by Street in \cite{street1987orientals} where he regards them as ``fundamental structures of nature".

Recall from (\ref{e:definition of W}) the definition of $W$, the minimal free resolution of $\Ftwo$ as an $\Ftwo[\Sigma_2]$-module.
A \textbf{counital cosymmetric coalgebra} is an augmented differential graded module $(C, \varepsilon)$ together with
\begin{equation*}
\Delta \colon W \otimes C \to C \otimes C
\end{equation*}
an $\Ftwo[\Sigma_2]$-linear chain map for which $\varepsilon$ acts as a counit.
We denote $\Delta(e_i \tensor c)$ by $\Delta_i(c)$.

We call $c \in C_n$ a \textbf{group-like element} if for any integer $k$ we have
\begin{equation*}
\Delta_k (c) \in C_{\leq n} \otimes C_{\leq n}
\end{equation*}
\begin{equation*}
\Delta_n(c) = c \otimes c
\end{equation*}
and, when $n = 0$,
\begin{equation*}
\varepsilon(c) = 1.
\end{equation*}

We say that $\big( C, \Delta , \varepsilon \big)$ is \textbf{group-like} if it admits a basis of group-like elements.

The chains of a simplicial set are equipped with a group-like counital cosymmetric coalgebra structure corresponding to Steenrod's cup-$i$ products
\begin{equation*}
(\alpha \smallsmile_{i} \beta)(c) = (\alpha \tensor \beta)\Delta_i(c).
\end{equation*}
With respect to this structure we have the following
\begin{theorem} [\cite{medina2020globular}]
	There exists a functor from group-like counital cosymmetric coalgebra to $n$-categories sending the chains on the standard $n$-simplex to $\mathcal O_n$.
\end{theorem}

The definition of this functor is similar to those used by Street, Brown, and Steiner in their respective studies of parity complexes \cite{street1991parity}, linear $\omega$-categories \cite{brown2003cubical}, and augmented directed complexes \cite{steiner2004omega}.

\subsection{An effective proof of the Cartan and Adem relations}

The celebrated square operations of Steenrod
\begin{equation*}
Sq^k \colon H^*(X; \mathbb F_2) \to H^*(X; \mathbb F_2)
\end{equation*}
are defined for a cohomology class $[\alpha] \in H^{-n}(X;\Ftwo)$ by
\begin{equation} \label{e:definition of squares in terms of cups}
Sq^k\big([\alpha]\big) = [\alpha \smallsmile_{k-n} \alpha].
\end{equation}
The axioms for the cup-$i$ products I presented in \cite{medina2018axiomatic} are inspired by the axioms of the square operations described in \cite{steenrod1962cohomology}
\begin{enumerate}
	\item $Sq^k$ is natural,
	\item $Sq^0$ is the identity,
	\item $Sq^k(x) = x^2$ for $x \in H^{-k}(X; \mathbb F_2)$,
	\item $Sq^k(x) = 0$ for $x \in H^{-n}(X; \mathbb F_2)$ with $n>k$,
	\item $Sq^k(xy) = \sum_{i+j=k} Sq^i (x) Sq^j(y)$.
\end{enumerate}
Axiom (5), known as the \textbf{Cartan formula}, is the focus of \cite{medina2020cartan}.
Using (\ref{e:definition of squares in terms of cups}), this formula is equivalent to
\begin{equation*}
0 =
\Big[ (\alpha \smallsmile_0 \beta) \smallsmile_i (\alpha \smallsmile_0 \beta)\ +
\sum_{i=j+k} (\alpha \smallsmile_j \alpha) \smallsmile_0 (\beta \smallsmile_k \beta) \Big].
\end{equation*}
In \cite{medina2020cartan}, I constructed effectively for any $i \geq 0$ and cocycles $\alpha, \beta \in N^*(X; \mathbb F_2)$ a natural cochain $\zeta_i(\alpha, \beta)$ such that
\begin{equation*}
\delta \zeta_i(\alpha, \beta) =
(\alpha \smallsmile_0 \beta) \smallsmile_i (\alpha \smallsmile_0 \beta)\ + \sum_{i=j+k} (\alpha \smallsmile_j \alpha) \smallsmile_0 (\beta \smallsmile_k \beta).
\end{equation*}

Similarly, one could try to produce these cochains for the Adem relations
\begin{equation*}
Sq^i Sq^j(\alpha) = \sum_{k=0}^{\lfloor i/2 \rfloor} {j-k-1 \choose i-2k} Sq^{i+j-k} Sq^k(\alpha).
\end{equation*}
New ideas are required for this, and it was carried through in \cite{medina2021adem} as joint work with G. Brumfiel and J. Morgan.

\subsection{Persistence Steenrod modules}

Persistence (co)homology is one of the main tools in the rapidly developing field of topological data analysis.
A motivating example for this technique is the study of discrete sets equipped with a metric, for example a point cloud of data inside euclidean space.
From it, we can construct a sequence of nested simplicial complexes
\begin{equation*}
X_1 \to X_2 \to \cdots \to X_n
\end{equation*}
and the naturality of the cohomology functor provides us with a sequence
\begin{equation} \label{e:mantequilla}
H^\bullet(X_1; \bF) \leftarrow H^\bullet(X_2; \bF) \leftarrow \cdots \leftarrow H^\bullet(X_n; \bF).
\end{equation}
Persistence cohomology focuses on the way the ranks of these vector spaces fit together.
More specifically, it identifies Betti numbers that are shared by many consecutive simplicial complexes.
These are regarded as topological features of the discrete metric space which are robust with respect to perturbations and noise.

When $\bF$ is the field with two elements $\Ftwo$ the sequence $(\ref{e:mantequilla})$ is equipped with a natural action of the Steenrod squares on each graded vector space.
This action can detect finer information beyond the Betti numbers.
The main contribution I made in \cite{medina2018persistence} is the development of the algorithms necessary to use this extra information in topological data analysis.

I am working in collaboration with the rest of \texttt{giotto-tda}'s development team to incorporate a high-performance version of my algorithms into our toolkit \cite{medina2021giotto}.