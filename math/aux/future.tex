
\section{Future work}

\subsection{Foundations of geometric (co)homology}

The use of manifold-like objects to represent (co)homology classes has a long history and many technical subtleties.
In \cite{medina2021foundations}, with G. Friedman and D. Sinha we further develop one such theory for smooth manifolds based on maps from manifolds with corners.
As an application of these foundations, we expect to extend our work relating transverse intersection and the cup product of cubical cochains (\cref{ss:flows}) to an entire $E_\infty$-structure.
More explicitly, we expect to introduce further coherent vector fields to model geometrically the $E_\infty$-structure defined in \cite{medina2021cubical}.

\subsection{Steenrod operations and Khovanov homology}

In work being finalized, with Federico Cantero-Mor\'{a}n we introduced a description of Steenrod operations at the simplicial cochain level that is dual to the one given in \cite{medina2020maysteenrod}.
There are two sources of motivation for this.
On one hand, for applications in persistent homology or computational topology more broadly, this description lends itself to a much faster algorithmic implementation, analogous to how the description in \cite{medina2021newformulas} of cup-$i$ products improved the performance of Steenrod square computations.
On the other, this dual description is applicable in more generality.
For example, the dual description of \cite{medina2021newformulas} was used by Cantero-Mor\'{a}n to define Steenrod squares on Khovanov homology, and we expect to define Steenrod operations at odd primes using our new description.

\subsection{Cochain level secondary operations}

Building on the work described in \cref{ss:cartan} and \cref{ss:adem} that constructs cochain level representatives enforcing the Cartan and Adem relations, we expect, using the formulas of \cite{medina2020maysteenrod}, to extend these constructions to all primes.
Effectively constructing cochains enforcing these relations and capable of representing secondary operations at the cochain level.

\subsection{Dyer-Lashof operations}

Configuration space of $r$ points in $\mathbb R^n$ are objects of central topological interest related to loop spaces, knots, and deformation theory.
Recently, they have also found applications in motion planning and robotics through the concept of topological complexity \cite{farber2003motion.planning}.
The homology of these spaces is a crucial invariant and we expect to identify chain level representatives for these classes on combinatorial models of the $E_n$-operads.
This work extends constructions in the $E_\infty$ case discussed in \cref{ss:may steenrod}.

\subsection{Representation and cohomology of categories}

The theory of representations and cohomology of groups can be recasted in terms of (Hopf) group rings.
Many of the results in this theory extend naturally to quivers and (small) categories but not all.
In particular, some important functoriality properties are lost in all current generalizations from groups to categories.
In work to appear, I argue that to prevent this loss one should consider the $E_2$-coalgebra structure carried by the chains on the nerve of the category.
When the category is a group, this corresponds to considering the $E_2$-coalgebra structure on the bar construction of the group algebra.

\subsection{Topology and deep learning}

Building on the work by \texttt{giotto-tda}'s team the following research direction will be pursued.
Topology can complement traditional approaches to machine learning and data analysis by providing global summaries of complex relational structures.
The added information is model-independent and highly resistant to noise.
Designing reliable AI systems is a multi-faceted challenge and topology can be used to tackle the following core aspects of the problem:
\begin{itemize}
	\item Robustness to noise and adversarial attacks: deep learning models are surprisingly susceptible to small perturbations, and this can lead to unexpected failures in deployed systems.
	\item Generalizability to unseen data: large models can overfit the data they are trained on and thus fail to generalize to data in deployment.
\end{itemize}
Traditional regularizers, one of the primary tools used to tackle the above challenges, impose penalties on representations that exhibit undesirable properties such as exploding gradients or highly complicated geometric structures.
By studying the topology of the weight space and decision boundaries of trained models, we will validate on real world data the assumption that robustness is linked to less cumbersome topologies.
This will enable to create performant topological regularizers that improve the resilience and generalization power of new models.

\subsection{$E_\infty$-algebras and L-theory}

With Manuel Rivera and Mahmoud Zeinalian we expect to combine the algebraic structures describing homotopy types \cite{sullivan1977infinitesimal, mandell2001padic} and those developed by Ranicki \cite{ranicki1992topological} for the classification of topological manifold structures.
Substantial partial results in this direction can be found in their paper \cite{rivera2018rigidification} and my thesis \cite{medina2015thesis}.

\subsection{Rational homotopy theory and $C_\infty$-coalgebras}

After the pioneering work of Quillen \cite{quillen1969rational} and Sullivan \cite{sullivan1977infinitesimal}, it is known that the categories of differential Lie and commutative (co)algebras model the rational homotopy category of spaces under certain finiteness assumptions.
Finite type is needed for Sullivan's approach whereas 1-connectedness is assumed for Quillen's.
To obtain a Lie model applicable for connected spaces with no finite type assumption it is desirable to develop a theory of $C_\infty$-coalgebras over $\mathbb{Q}$.
Using my model of the $E_\infty$ operad I expect to describe such theory in the future.

\subsection{Symmetries of the associahedral diagonal}

Defining cellular diagonals of polytopes is a problem with combinatorial interest.
In the case of Stasheff associahedra, polytopes controling $A_\infty$-algebras,
the existence of a diagonal ensures that the category of $A_\infty$-algebras is equipped with a well defined tensor product.
This structure is key for applications in Heegard-Floer homology and other areas of symplectic geometry via Fukaya categories \cite{lipshitz2020diagonals}.

Using my model of the $E_\infty$-operad, I expect to introduce a family of coherent homotopies correcting the lack of symmetry of the associahedral diagonal and other polytopes of interest.
